\documentclass[a4paper,10pt]{article}

\usepackage[francais]{babel}
\usepackage[utf8]{inputenc}
\usepackage[T1]{fontenc}
\usepackage{amsmath}
\usepackage{graphicx}
\usepackage{eurosym}
\usepackage[cdot]{SIunits}
\usepackage{tikz}
\usetikzlibrary{arrows}
\usepackage[europeanresistors]{circuitikz}
\usepackage{chemfig}
\usepackage{fancyhdr}
\usepackage{multido}
\usepackage{geometry}
\usepackage{titlesec}
\usepackage{multicol}
\usepackage{multirow}
\usepackage{enumitem}

\usepackage{colorwav}
\usepackage{etoolbox} 
\renewcommand*{\do}[1]{ \storeRGBofWavelength{\Rval}{\Gval}{\Bval}{#1} \definecolor{wl#1}{rgb}{\Rval, \Gval, \Bval} } \docsvlist{380,390,400,410,420,430,440,450,500,550,600,650,700,750,780}
\pgfdeclarehorizontalshading{spectrum}{1cm}{ color(0cm)=(wl380); color({0.4cm})=(wl400); color(1.4cm)=(wl450); color(2.4cm)=(wl500); color(3.4cm)=(wl550); color(4.4cm)=(wl600); color(5.4cm)=(wl650); color(6.4cm)=(wl700); color(7.4cm)=(wl750); color(8cm)=(wl780) }
\usepackage{pgfplots}
\usetikzlibrary{plotmarks}
\pgfplotsset{compat=1.15}

\setlength{\parindent}{0mm}

\geometry{%
	a4paper,
	body={180mm,255mm},
	left=15mm,top=25mm,
	headheight=10mm,headsep=4mm,
	marginparsep=5mm,
	marginparwidth=40mm
}
\setlength{\headwidth}{180mm}

\renewcommand \thesection { }

\lhead{Mr Fontaine\\Lycée Jean Joly}
\chead{Activité expérimentale 14\\Degré d'acidité d'un vinaigre}
\rhead{Terminale S\\2018-2019}

\begin{document}
\thispagestyle{fancy}
\section{ANALYSER}
\begin{enumerate}
    \item Le degré du vinaigre est 8\%.
    
    La concentration molaire est $c=\frac{n_{acide}}{V_{vinaigre}}$.
    
    On calcule $n_{acide}$ en faisant $n_{acide}=\frac{m_{acide}}{M_{acide}}$ avec $m_{acide}=\unit{8}{\gram}$ et $M_{acide}=\unit{60}{\gram\usk\reciprocal\mole}$. 
    
    D'autre part, $V_{vinaigre}=\frac{m_{vinaigre}}{\rho_{vinaigre}}$ avec $m_{vinaigre}=\unit{100}{\gram}$ et $\rho_{vinaigre}=\unit{1,01\times 10^3}{\gram\usk\reciprocal\liter}$.
    
    Donc
    \[
    c=\frac{m_{acide}}{M_{acide}}\times \frac{\rho_{vinaigre}}{m_{vinaigre}}=\frac{8\times 1,01\times 10^3}{60\times 100}=\unit{1,35}{\mole\usk\reciprocal\liter}
    \]
    \item On réalise une dilution:
    
    \begin{tabular}{p{5cm}|p{5cm}|p{5cm}}
        \textbf{Première étape: } On prélève $\unit{10}{\milli\liter}$ de vinaigre à l'aide d'une pipette jaugée. &  \textbf{Deuxième étape: }On verse les $\unit{10}{\milli\liter}$ prélevés dans une fiole jaugée de $\unit{100}{\milli\liter}$.&\textbf{Troisième étape: }On complète avec de l'eau distillée.\\
            \def\svgwidth{3cm}
            \input{pipette.pdf_tex}
        & 
            \def\svgwidth{4cm}
            \input{fiole.pdf_tex}
        & 
            \def\svgwidth{4.5cm}
            \input{pissette.pdf_tex}
       \\
    \end{tabular}
\end{enumerate}
\section{RÉALISER}
On réalise un premier dosage pour déterminer l'intervalle dans lequel se trouve l'équivalence. Il se situe entre 12 et 14 mL.

Dans cette zone, on observe un changement de couleur dû à la présence de la phénolphtaléine.

\def\imagetop#1{\vtop{\null\hbox{#1}}}
\begin{tabular}{p{4cm} p{10cm}}

\imagetop{\scalebox{0.8 }{ 
\begin{tabular}{|c|c|}
 V(mL) & pH\\ \hline
 0,000 & 2,800\\
 1,000 & 3,500\\
2,000 & 3,800\\
 3,000 & 4,000\\
 4,000 & 4,200\\
 5,000 & 4,300 \\
 6,000 & 4,400\\
 7,000 & 4,600 \\
 8,000 & 4,700 \\
 9,000 & 4,900\\
 10,00 & 5,000\\
 11,00 & 5,200\\
 12,00 & 5,500\\
 12,20 & 5,600 \\
 12,40 & 5,700 \\
 12,60 & 5,800\\
 12,80 & 6,100\\
 13,00 & 6,400\\
 13,20 & 8,200\\
 13,40 & 9,600\\
 13,60 & 10,40 \\
 13,80 & 10,70 \\
 14,00 & 11,00\\
 15,00 & 11,30\\
 16,00 & 11,50 \\
 17,00 & 11,60 \\
 18,00 & 11,60 \\
 19,00 & 11,70\\
 20,00 & 11,80\\
\end{tabular}}}
&
\imagetop{\begin{tikzpicture}
    \begin{axis}[
        height=8cm,width=12cm
        ,axis x line=bottom,axis y line=left
        ,xmin=0,xmax=20.8
        ,ymin=0,ymax=12.27
        ,grid=major
        ,title={}
        ,xlabel={V(mL)}
        ,ylabel={pH}
        ]
        \addplot[draw=black,mark=+] file {graphe10.txt};
    \end{axis}
\end{tikzpicture}}
\\
\end{tabular}
\section{VALIDER}
\begin{enumerate}
    \item L'équation de la réaction de titrage est :
    \begin{center}
        \chemfig{CH_3COOH}+\chemfig{HO^{-}} $\longrightarrow$ \chemfig{CH_3COO^{-}} + \chemfig{H_2O}
    \end{center}
    La réaction de titrage est totale et à l'équivalence, la soude a entièrement consommé l'acide présent à l'état initial.
    
    On en déduit qu'à l'équivalence, la quantité de matière de soude introduite est égale à la quantité de matière d'acide présent dans le bécher à l'état initial:
    \[
        n_{BE}=m{Ai}
    \]
    \item 
\end{enumerate}
\end{document}
